\documentclass{beamer}
\usepackage[size=a1,orientation=portrait,scale=1.8]{beamerposter}
\usetheme{LLT-poster}
\usecolortheme{ComingClean}
\usepackage[utf8]{inputenc}
\usepackage[T1]{fontenc}
\usepackage{libertine}
\usepackage[scaled=0.92]{inconsolata}
\usepackage[libertine]{newtxmath}
\usepackage[subpreambles=true]{standalone}
\author[a.savagar@kent.ac.uk]{Anthony Savagar}
\title{UK Firm Creation During the Covid-19 Pandemic \\ \small Alfred Duncan, Miguel León-Ledesma, Anthony Savagar}
\institute{University of Kent}

\begin{document}
  \begin{frame}{} 
    \begin{block}{\large Main Messages}
        \begin{itemize}
            \item \alert{Aim:} Measure UK firm creation during Covid-19 pandemic to quantify supply-side disruption in real time. 
            \item Firm creation fell sharply during March, April and May but recovered strongly during June and July.
            \item Firm creation is important for productivity, wages and employment in the long run.
        \end{itemize}
    \end{block}
    \begin{block}{\large Data}
        \begin{itemize}
        \item The UK company register is available from Companies House. It records every company registered in the UK.
        \item The register is released monthly which allows us to record daily registrations over the previous month. 
        \item We compare the monthly register release in 2020 with the equivalent monthly release in 2019.
            \item \alert{Advantages:} Real-time; population-wide; detailed company location and sector; high-frequency (daily).
            \item \alert{Disadvantages:} Few variables; short time series; death poorly measured; records all companies many will not produce.
        \end{itemize}
    \end{block}
    \begin{columns}[t]
      \begin{column}{.49\linewidth}
        \begin{block}{Aggregate Plot}
        \documentclass[]{standalone}
\usepackage{pgfplots,
            pgfplotstable,
            }
\pgfplotsset{compat=newest}
\usepgfplotslibrary{dateplot, statistics}

\usepackage{amsmath}

\begin{document}

\begin{tikzpicture}
    \begin{axis}[clip=false,
                ylabel style={align=center}, 
                ylabel=,
                legend pos= north west,
                /pgf/number format/1000 sep={},
                axis lines=left, 
                grid = both,
                legend style={
                at={(0.1,1.3)},
                draw=none,
                fill=white,
                legend columns=2,
                },
                width=0.9\textwidth,
                height=9cm,
                date coordinates in=x,
                xticklabel={\day/\month},
                xtick={2020-01-14, 2020-03-17, 2020-05-19, 2020-07-28},
                xticklabel style={rotate=0, anchor=near xticklabel}
                ]
        \addplot+[no markers, thick, black] table[x=date, y=births,col sep=comma] {company_registrations_against_2019.csv};
    \end{axis}
\end{tikzpicture}
\end{document}
        \end{block}

        \begin{block}{Region Plot}
        \documentclass[]{standalone}
\usepackage{pgfplots,
            pgfplotstable,
            }
\pgfplotsset{compat=newest}
\usepgfplotslibrary{dateplot, statistics}


\usepackage{amsmath}

\begin{document}

\begin{tikzpicture}

    \begin{axis}[clip=false,
                very thick,
                /pgf/number format/1000 sep={},
                axis lines=left, 
                grid = both,
                width=0.9\textwidth,
                height=10cm,
                date coordinates in=x,
                xticklabel={\day/\month},
                xtick={2020-04-06, 2020-05-04, 2020-06-01, 2020-06-29, 2020-07-27},
                xticklabel style={rotate=0, anchor=near xticklabel}
                ]
        \addplot+[no markers, black] table[x=date, y=Wales,col sep=comma] {lockdown_fall_by_coarse_region_ts.csv} node[above,pos=1] {Wal};

        \addplot+[dotted, ultra thick, no markers] table[x=date, y=Northern Ireland,col sep=comma] {lockdown_fall_by_coarse_region_ts.csv} node[below,pos=1] {N.Ir};

        \addplot+[no markers, dashed] table[x=date, y=England excl London,col sep=comma] {lockdown_fall_by_coarse_region_ts.csv} node[above,pos=1] {Eng};

        \addplot+[no markers] table[x=date, y=London,col sep=comma] {lockdown_fall_by_coarse_region_ts.csv} node[above,pos=1] {Lon};

        \addplot+[no markers] table[x=date, y=Scotland,col sep=comma] {lockdown_fall_by_coarse_region_ts.csv} node[above,pos=1] {Sco};
    \end{axis}
\end{tikzpicture}
\end{document}
        \end{block}
        
        \begin{block}{Sector Plot}
        \documentclass[]{standalone}
\usepackage{pgfplots,
            pgfplotstable,
            }
\pgfplotsset{compat=newest}
\usepgfplotslibrary{dateplot, statistics}
\usepackage{amsmath}

\begin{document}
\begin{tikzpicture}
    \begin{axis}[
                clip = false,
                ylabel style={align=center}, 
                ylabel=,
                legend pos= north west,
                very thick,
                /pgf/number format/1000 sep={},
                axis lines=left, 
                grid = both,
                legend style={
                        at={(0.1,1.3)},
                        draw=none,
                        fill=white,
                        legend columns=2,
                        },
                width=0.9\textwidth,
                height=10cm,
                date coordinates in=x,
                xticklabel={\day/\month},
                xtick={2020-04-06, 2020-05-04, 2020-06-01, 2020-06-29, 2020-07-27},
                xticklabel style={rotate=0, anchor=near xticklabel}
                ]
        \addplot+[no markers, dashed] table[x=date, y=Construction,col sep=comma] {company_registrations_against_2019_by_selected_industry_lines.csv} node[below left,pos=1] {Construction};

        \addplot+[mark=none] table[x=date,
        y=Information and Communication,col sep=comma] 
        {company_registrations_against_2019_by_selected_industry_lines.csv} node[above left,pos=1] {Info \& Comms};

        \addplot+[mark=none] table[x=date,
        y=Public administration and defence and social security,col sep=comma] 
        {company_registrations_against_2019_by_selected_industry_lines.csv} node[above left,pos=1] {Public ad; Def; Soc sec};

        \addplot+[no markers] table[x=date, y=Wholesale and retail,col sep=comma] {company_registrations_against_2019_by_selected_industry_lines.csv} node[above left,pos=1] {Wholesale \& Retail};
    \end{axis}
\end{tikzpicture}
\end{document}
      \end{block}
      \end{column}
      
      \begin{column}{.49\linewidth}
        \begin{block}{Aggregate Comment}
          \begin{itemize}
          \item Weekly business creation relative to 2019 (= 100). 
          \item \alert{Timeline:} 23/03 lockdown begins. 11/05 re-opening begins.
          \item There are three clear periods:
          \begin{itemize}
              \item \alert{Pre-lockdown:} weekly firm creation is equal to 2019. 
              \item \alert{Severe lockdown:} Firm creation falls sharply in late March and remains low until late May.
              \item \alert{Easing lockdown:} Firm creation exceeds its 2019 level consistently after late May.
          \end{itemize}
          \end{itemize}
\vspace{0.3cm}
        \end{block}

        \begin{block}{Region Comment}
          \begin{itemize}
          \item Cumulative business creation from March 23 (lockdown began) relative to the same period in 2019.
          \item Business creation between 23 Mar - 06 Apr was near 50\% less for all countries than over the same period in 2019. 
          \item Business creation between 23 Mar - 27 Jul was 10\% higher in London, 5\% lower in England (excl. Lon) and 20-30\% lower in Wales, Scotland and N Ireland.
          \end{itemize}
\vspace{2.4cm}
        \end{block}
        
        \begin{block}{Sector Comment}
          \begin{itemize}
          \item Cumulative business creation from March 23 (lockdown began) relative to the same period in 2019.
          \item Business creation between 23 Mar - 06 Apr was near 30\% less in most sectors and 50\% less in construction than over the same period in 2019.
          \item Business creation between 23 Mar - 27 Jul was 50\% higher in wholesale \& retail and 20\% lower in construction relative to 2019.
          \end{itemize}
\vspace{1.25cm}
        \end{block}

      \end{column}
    \end{columns}
        \begin{block}{\large Implications}
        \begin{itemize}
            \item Firm creation has long-run effects on employment, productivity and growth.
            \item From US evidence we know:
            \begin{itemize}
                \item New firms are net job creators: they create more jobs than they destroy.
                \item New firms drive future productivity through their own innovation and through competitive effects on incumbents.
            \end{itemize}  
        \end{itemize}
    \end{block}
  \end{frame}
\end{document}